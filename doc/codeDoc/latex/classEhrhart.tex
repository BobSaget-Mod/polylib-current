\section{Ehrhart  Class Reference}
\label{classEhrhart}\index{Ehrhart@{Ehrhart}}


\subsection{Detailed Description}
The following are mainly for debug purposes. You shouldn't need to change anything for daily usage...





you may define each macro independently  \begin{enumerate}
\item 
 define EDEBUG minimal debug  \item 
 define EDEBUG1 prints enumeration points \item 
 define EDEBUG11 prints number of points \item 
 define EDEBUG2 prints domains \item 
 define EDEBUG21 prints more domains \item 
 define EDEBUG3 prints systems of equations that are solved \item 
 define EDEBUG4 prints message for degree reduction \item 
 define EDEBUG5 prints result before simplification  \item 
 define EDEBUG6 prints domains in Preprocess  \item 
 define EDEBUG61 prints even more in Preprocess \item 
 define EDEBUG62 prints domains in Preprocess2 \end{enumerate}




The documentation for this class was generated from the following file:\begin{CompactItemize}
\item 
{\bf ehrhart.c}\end{CompactItemize}
