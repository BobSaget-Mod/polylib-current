\section{ehrhart.c File Reference}
\label{ehrhart_8c}\index{ehrhart.c@{ehrhart.c}}
{\tt \#include $<$stdio.h$>$}\par
{\tt \#include $<$stdlib.h$>$}\par
{\tt \#include $<$ctype.h$>$}\par
{\tt \#include $<$string.h$>$}\par
{\tt \#include $<$unistd.h$>$}\par
{\tt \#include $<$assert.h$>$}\par
{\tt \#include $<$polylib/polylib.h$>$}\par
\subsection*{Defines}
\begin{CompactItemize}
\item 
\#define {\bf EPRINT\_\-ALL\_\-VALIDITY\_\-CONSTRAINTS}
\begin{CompactList}\small\item\em define this to print all constraints on the validity domains if not defined, only new constraints (not in validity domain given by the user) are printed.\item\end{CompactList}\item 
\#define {\bf REDUCE\_\-DEGREE}
\begin{CompactList}\small\item\em Reduce the degree of resulting polynomials.\item\end{CompactList}\item 
\#define {\bf ALL\_\-OVERFLOW\_\-WARNINGS}
\begin{CompactList}\small\item\em define this to print one warning message per domain overflow these overflows should no longer happen since version 4.20.\item\end{CompactList}\item 
\#define {\bf EPRINT}
\item 
\#define {\bf MAXITER}\ 100
\begin{CompactList}\small\item\em This procedure finds an integer point contained in polyhedron D / first checks for positive values, then for negative values returns TRUE on success.\item\end{CompactList}\end{CompactItemize}
\subsection*{Functions}
\begin{CompactItemize}
\item 
enode $\ast$ {\bf new\_\-enode} (enode\_\-type type, int size, int pos)
\begin{CompactList}\small\item\em EHRHART POLYNOMIAL SYMBOLIC ALGEBRA SYSTEM.\item\end{CompactList}\item 
void {\bf free\_\-evalue\_\-refs} (evalue $\ast$e)
\begin{CompactList}\small\item\em releases all memory referenced by e.\item\end{CompactList}\item 
enode $\ast$ {\bf ecopy} (enode $\ast$e)
\item 
void {\bf print\_\-evalue} (FILE $\ast$DST, evalue $\ast$e, char $\ast$$\ast$pname)
\item 
void {\bf print\_\-enode} (FILE $\ast$DST, enode $\ast$p, char $\ast$$\ast$pname)
\begin{CompactList}\small\item\em prints the enode to DST.\item\end{CompactList}\item 
int {\bf eequal} (evalue $\ast$e1, evalue $\ast$e2)
\item 
void {\bf reduce\_\-evalue} (evalue $\ast$e)
\item 
void {\bf emul} (evalue $\ast$e1, evalue $\ast$e2, evalue $\ast$res)
\begin{CompactList}\small\item\em multiplies two evalues and puts the result in res.\item\end{CompactList}\item 
void {\bf eadd} (evalue $\ast$e1, evalue $\ast$res)
\begin{CompactList}\small\item\em adds one evalue to evalue 'res.\item\end{CompactList}\item 
void {\bf edot} (enode $\ast$v1, enode $\ast$v2, evalue $\ast$res)
\begin{CompactList}\small\item\em computes the inner product of two vectors.\item\end{CompactList}\item 
void {\bf aep\_\-evalue} (evalue $\ast$e, int $\ast$ref)
\begin{CompactList}\small\item\em local recursive function used in the following ref contains the new position for each old index position.\item\end{CompactList}\item 
void {\bf addeliminatedparams\_\-evalue} (evalue $\ast$e, Matrix $\ast$CT)
\begin{CompactList}\small\item\em Comments.\item\end{CompactList}\item 
int {\bf cherche\_\-min} (Value $\ast$min, Polyhedron $\ast$D, int pos)
\item 
Polyhedron $\ast$ {\bf Polyhedron\_\-Preprocess} (Polyhedron $\ast$D, Value $\ast$size, unsigned MAXRAYS)
\begin{CompactList}\small\item\em This procedure finds the smallest parallelepiped of size '{\em size[i]}' for every dimension i, contained in polyhedron D If this is not possible, an empty polyhedron is returned.\item\end{CompactList}\item 
Polyhedron $\ast$ {\bf Polyhedron\_\-Preprocess2} (Polyhedron $\ast$D, Value $\ast$size, Value $\ast$lcm, unsigned MAXRAYS)
\begin{CompactList}\small\item\em This procedure finds an hypercube of size 'size', containing polyhedron D increases size and lcm if necessary (and not \char`\"{}too big\char`\"{}) If this is not possible, an empty polyhedron is returned.\item\end{CompactList}\item 
Polyhedron $\ast$ {\bf old\_\-Polyhedron\_\-Preprocess} (Polyhedron $\ast$D, Value size, unsigned MAXRAYS)
\begin{CompactList}\small\item\em This procedure adds additional constraints to D so that as each parameter is scanned, it will have a minimum of 'size' points If this is not possible, an empty polyhedron is returned.\item\end{CompactList}\item 
int {\bf count\_\-points} (int pos, Polyhedron $\ast$P, Value $\ast$context)
\begin{CompactList}\small\item\em PROCEDURES TO COMPUTE ENUMERATION.\item\end{CompactList}\item 
enode $\ast$ {\bf P\_\-Enum} (Polyhedron $\ast$L, Polyhedron $\ast$LQ, Value $\ast$context, int pos, int nb\_\-param, int dim, Value $\ast$lcm)
\item 
void {\bf Scan\_\-Vertices} (Param\_\-Polyhedron $\ast$PP, Param\_\-Domain $\ast$Q, Matrix $\ast$CT, Value $\ast$lcm, int nbp)
\item 
Enumeration $\ast$ {\bf Enumerate\_\-No\-Parameters} (Polyhedron $\ast$P, Polyhedron $\ast$C, Matrix $\ast$CT, Polyhedron $\ast$CEq, unsigned MAXRAYS)
\begin{CompactList}\small\item\em Procedure to count points in a non-parameterized polytope.\item\end{CompactList}\item 
Enumeration $\ast$ {\bf Polyhedron\_\-Enumerate} (Polyhedron $\ast${\bf Pi}, Polyhedron $\ast$C, unsigned MAXRAYS)
\begin{CompactList}\small\item\em Procedure to count points in a parameterized polytope.\item\end{CompactList}\end{CompactItemize}
\subsection*{Variables}
\begin{CompactItemize}
\item 
int {\bf overflow\_\-warning\_\-flag} = 1
\begin{CompactList}\small\item\em EPRINT : print results while computing the ehrhart polynomial.\item\end{CompactList}\item 
char $\ast$$\ast$ {\bf param\_\-name}
\end{CompactItemize}


\subsection{Define Documentation}
\index{ehrhart.c@{ehrhart.c}!ALL_OVERFLOW_WARNINGS@{ALL\_\-OVERFLOW\_\-WARNINGS}}
\index{ALL_OVERFLOW_WARNINGS@{ALL\_\-OVERFLOW\_\-WARNINGS}!ehrhart.c@{ehrhart.c}}
\subsubsection{\setlength{\rightskip}{0pt plus 5cm}\#define ALL\_\-OVERFLOW\_\-WARNINGS}\label{ehrhart_8c_a2}


define this to print one warning message per domain overflow these overflows should no longer happen since version 4.20.



Definition at line 76 of file ehrhart.c.\index{ehrhart.c@{ehrhart.c}!EPRINT@{EPRINT}}
\index{EPRINT@{EPRINT}!ehrhart.c@{ehrhart.c}}
\subsubsection{\setlength{\rightskip}{0pt plus 5cm}\#define EPRINT}\label{ehrhart_8c_a3}




Definition at line 101 of file ehrhart.c.\index{ehrhart.c@{ehrhart.c}!EPRINT_ALL_VALIDITY_CONSTRAINTS@{EPRINT\_\-ALL\_\-VALIDITY\_\-CONSTRAINTS}}
\index{EPRINT_ALL_VALIDITY_CONSTRAINTS@{EPRINT\_\-ALL\_\-VALIDITY\_\-CONSTRAINTS}!ehrhart.c@{ehrhart.c}}
\subsubsection{\setlength{\rightskip}{0pt plus 5cm}\#define EPRINT\_\-ALL\_\-VALIDITY\_\-CONSTRAINTS}\label{ehrhart_8c_a0}


define this to print all constraints on the validity domains if not defined, only new constraints (not in validity domain given by the user) are printed.



Definition at line 52 of file ehrhart.c.\index{ehrhart.c@{ehrhart.c}!MAXITER@{MAXITER}}
\index{MAXITER@{MAXITER}!ehrhart.c@{ehrhart.c}}
\subsubsection{\setlength{\rightskip}{0pt plus 5cm}\#define MAXITER\ 100}\label{ehrhart_8c_a4}


This procedure finds an integer point contained in polyhedron D / first checks for positive values, then for negative values returns TRUE on success.

Result is in min. returns FALSE if no integer point is found





This is the maximum number of iterations for a given parameter to find a integer point inside the context. Kind of weird. cherche\_\-min should



\begin{Desc}
\item[Parameters: ]\par
\begin{description}
\item[{\em 
min}]\item[{\em 
D}]\item[{\em 
pos}]\end{description}
\end{Desc}


Definition at line 638 of file ehrhart.c.\index{ehrhart.c@{ehrhart.c}!REDUCE_DEGREE@{REDUCE\_\-DEGREE}}
\index{REDUCE_DEGREE@{REDUCE\_\-DEGREE}!ehrhart.c@{ehrhart.c}}
\subsubsection{\setlength{\rightskip}{0pt plus 5cm}\#define REDUCE\_\-DEGREE}\label{ehrhart_8c_a1}


Reduce the degree of resulting polynomials.



Definition at line 70 of file ehrhart.c.

\subsection{Function Documentation}
\index{ehrhart.c@{ehrhart.c}!Enumerate_NoParameters@{Enumerate\_\-NoParameters}}
\index{Enumerate_NoParameters@{Enumerate\_\-NoParameters}!ehrhart.c@{ehrhart.c}}
\subsubsection{\setlength{\rightskip}{0pt plus 5cm}Enumeration$\ast$ Enumerate\_\-No\-Parameters (Polyhedron $\ast$ {\em P}, Polyhedron $\ast$ {\em C}, Matrix $\ast$ {\em CT}, Polyhedron $\ast$ {\em CEq}, unsigned {\em MAXRAYS})}\label{ehrhart_8c_a26}


Procedure to count points in a non-parameterized polytope.

\begin{Desc}
\item[Parameters: ]\par
\begin{description}
\item[{\em 
P}]Polyhedron to count \item[{\em 
C}]Parameter Context domain \item[{\em 
CT}]Matrix to transform context to original \end{description}
\end{Desc}
\begin{Desc}
\item[ma CEq     additionnal equalities in context]\par
\end{Desc}
\begin{Desc}
\item[Parameters: ]\par
\begin{description}
\item[{\em 
MAXRAYS}]workspace size \end{description}
\end{Desc}


Definition at line 1705 of file ehrhart.c.\index{ehrhart.c@{ehrhart.c}!P_Enum@{P\_\-Enum}}
\index{P_Enum@{P\_\-Enum}!ehrhart.c@{ehrhart.c}}
\subsubsection{\setlength{\rightskip}{0pt plus 5cm}enode$\ast$ P\_\-Enum (Polyhedron $\ast$ {\em L}, Polyhedron $\ast$ {\em LQ}, Value $\ast$ {\em context}, int {\em pos}, int {\em nb\_\-param}, int {\em dim}, Value $\ast$ {\em lcm})\hspace{0.3cm}{\tt  [static]}}\label{ehrhart_8c_a24}




Definition at line 1236 of file ehrhart.c.\index{ehrhart.c@{ehrhart.c}!Polyhedron_Enumerate@{Polyhedron\_\-Enumerate}}
\index{Polyhedron_Enumerate@{Polyhedron\_\-Enumerate}!ehrhart.c@{ehrhart.c}}
\subsubsection{\setlength{\rightskip}{0pt plus 5cm}Enumeration$\ast$ Polyhedron\_\-Enumerate (Polyhedron $\ast$ {\em Pi}, Polyhedron $\ast$ {\em C}, unsigned {\em MAXRAYS})}\label{ehrhart_8c_a27}


Procedure to count points in a parameterized polytope.

\begin{Desc}
\item[Parameters: ]\par
\begin{description}
\item[{\em 
P}]Polyhedron to enumerate \item[{\em 
C}]Context Domain \item[{\em 
MAXRAYS}]size of workspace \end{description}
\end{Desc}
\begin{Desc}
\item[Returns: ]\par
a list of validity domains + evalues EP \end{Desc}


Definition at line 1813 of file ehrhart.c.\index{ehrhart.c@{ehrhart.c}!Polyhedron_Preprocess@{Polyhedron\_\-Preprocess}}
\index{Polyhedron_Preprocess@{Polyhedron\_\-Preprocess}!ehrhart.c@{ehrhart.c}}
\subsubsection{\setlength{\rightskip}{0pt plus 5cm}Polyhedron$\ast$ Polyhedron\_\-Preprocess (Polyhedron $\ast$ {\em D}, Value $\ast$ {\em size}, unsigned {\em MAXRAYS})}\label{ehrhart_8c_a20}


This procedure finds the smallest parallelepiped of size '{\em size[i]}' for every dimension i, contained in polyhedron D If this is not possible, an empty polyhedron is returned.





\small\begin{alltt}
written by vin100, 2000, for version 4.19
modified 2002, version 5.10
\end{alltt}\normalsize 






It first finds the coordinates of the lexicographically smallest edge of the hypercube, obtained by transforming the constraints of D (by adding 'size' as many times as there are negative coeficients in each constraint), and finding the lexicographical min of this polyhedron. Then it builds the hypercube and returns it.



 \begin{Desc}
\item[Parameters: ]\par
\begin{description}
\item[{\em 
D}]\item[{\em 
size}]\item[{\em 
MAXRAYS}]\end{description}
\end{Desc}


Definition at line 747 of file ehrhart.c.\index{ehrhart.c@{ehrhart.c}!Polyhedron_Preprocess2@{Polyhedron\_\-Preprocess2}}
\index{Polyhedron_Preprocess2@{Polyhedron\_\-Preprocess2}!ehrhart.c@{ehrhart.c}}
\subsubsection{\setlength{\rightskip}{0pt plus 5cm}Polyhedron$\ast$ Polyhedron\_\-Preprocess2 (Polyhedron $\ast$ {\em D}, Value $\ast$ {\em size}, Value $\ast$ {\em lcm}, unsigned {\em MAXRAYS})}\label{ehrhart_8c_a21}


This procedure finds an hypercube of size 'size', containing polyhedron D increases size and lcm if necessary (and not \char`\"{}too big\char`\"{}) If this is not possible, an empty polyhedron is returned.





\small\begin{alltt} written by vin100, 2001, for version 4.19\end{alltt}\normalsize 
\begin{Desc}
\item[Parameters: ]\par
\begin{description}
\item[{\em 
D}]\item[{\em 
size}]\item[{\em 
lcm}]\item[{\em 
MAXRAYS}]\end{description}
\end{Desc}


Definition at line 902 of file ehrhart.c.\index{ehrhart.c@{ehrhart.c}!Scan_Vertices@{Scan\_\-Vertices}}
\index{Scan_Vertices@{Scan\_\-Vertices}!ehrhart.c@{ehrhart.c}}
\subsubsection{\setlength{\rightskip}{0pt plus 5cm}void Scan\_\-Vertices (Param\_\-Polyhedron $\ast$ {\em PP}, Param\_\-Domain $\ast$ {\em Q}, Matrix $\ast$ {\em CT}, Value $\ast$ {\em lcm}, int {\em nbp})\hspace{0.3cm}{\tt  [static]}}\label{ehrhart_8c_a25}




Definition at line 1629 of file ehrhart.c.\index{ehrhart.c@{ehrhart.c}!addeliminatedparams_evalue@{addeliminatedparams\_\-evalue}}
\index{addeliminatedparams_evalue@{addeliminatedparams\_\-evalue}!ehrhart.c@{ehrhart.c}}
\subsubsection{\setlength{\rightskip}{0pt plus 5cm}void addeliminatedparams\_\-evalue (evalue $\ast$ {\em e}, Matrix $\ast$ {\em CT})\hspace{0.3cm}{\tt  [static]}}\label{ehrhart_8c_a18}


Comments.



Definition at line 592 of file ehrhart.c.\index{ehrhart.c@{ehrhart.c}!aep_evalue@{aep\_\-evalue}}
\index{aep_evalue@{aep\_\-evalue}!ehrhart.c@{ehrhart.c}}
\subsubsection{\setlength{\rightskip}{0pt plus 5cm}void aep\_\-evalue (evalue $\ast$ {\em e}, int $\ast$ {\em ref})\hspace{0.3cm}{\tt  [static]}}\label{ehrhart_8c_a17}


local recursive function used in the following ref contains the new position for each old index position.

\begin{Desc}
\item[Parameters: ]\par
\begin{description}
\item[{\em 
e}]pointer to an evalue \item[{\em 
ref}]transformation Matrix \end{description}
\end{Desc}


Definition at line 572 of file ehrhart.c.\index{ehrhart.c@{ehrhart.c}!cherche_min@{cherche\_\-min}}
\index{cherche_min@{cherche\_\-min}!ehrhart.c@{ehrhart.c}}
\subsubsection{\setlength{\rightskip}{0pt plus 5cm}int cherche\_\-min (Value $\ast$ {\em min}, Polyhedron $\ast$ {\em D}, int {\em pos})}\label{ehrhart_8c_a19}




Definition at line 639 of file ehrhart.c.\index{ehrhart.c@{ehrhart.c}!count_points@{count\_\-points}}
\index{count_points@{count\_\-points}!ehrhart.c@{ehrhart.c}}
\subsubsection{\setlength{\rightskip}{0pt plus 5cm}int count\_\-points (int {\em pos}, Polyhedron $\ast$ {\em P}, Value $\ast$ {\em context})}\label{ehrhart_8c_a23}


PROCEDURES TO COMPUTE ENUMERATION.

recursive procedure, recurse for each imbriquation\begin{Desc}
\item[Parameters: ]\par
\begin{description}
\item[{\em 
pos}]index position of current loop index (1..hdim-1) \item[{\em 
P}]loop domain \item[{\em 
context}]context values for fixed indices  \end{description}
\end{Desc}
\begin{Desc}
\item[Returns: ]\par
the number of integer points in this polyhedron \end{Desc}


Definition at line 1145 of file ehrhart.c.\index{ehrhart.c@{ehrhart.c}!eadd@{eadd}}
\index{eadd@{eadd}!ehrhart.c@{ehrhart.c}}
\subsubsection{\setlength{\rightskip}{0pt plus 5cm}void eadd (evalue $\ast$ {\em e1}, evalue $\ast$ {\em res})}\label{ehrhart_8c_a15}


adds one evalue to evalue 'res.

result = res + e1\begin{Desc}
\item[Parameters: ]\par
\begin{description}
\item[{\em 
e1}]an evalue  \item[{\em 
res}]\end{description}
\end{Desc}


Definition at line 413 of file ehrhart.c.\index{ehrhart.c@{ehrhart.c}!ecopy@{ecopy}}
\index{ecopy@{ecopy}!ehrhart.c@{ehrhart.c}}
\subsubsection{\setlength{\rightskip}{0pt plus 5cm}enode$\ast$ ecopy (enode $\ast$ {\em e})}\label{ehrhart_8c_a9}


\begin{Desc}
\item[Parameters: ]\par
\begin{description}
\item[{\em 
e}]pointer to an evalue  \end{description}
\end{Desc}
\begin{Desc}
\item[Returns: ]\par
description \end{Desc}


Definition at line 174 of file ehrhart.c.\index{ehrhart.c@{ehrhart.c}!edot@{edot}}
\index{edot@{edot}!ehrhart.c@{ehrhart.c}}
\subsubsection{\setlength{\rightskip}{0pt plus 5cm}void edot (enode $\ast$ {\em v1}, enode $\ast$ {\em v2}, evalue $\ast$ {\em res})}\label{ehrhart_8c_a16}


computes the inner product of two vectors.

Result = result (evalue) = v1.v2 (dot product)\begin{Desc}
\item[Parameters: ]\par
\begin{description}
\item[{\em 
v1}]an enode (vector) \item[{\em 
v2}]an enode (vector of constants) \item[{\em 
res}]result (evalue) \end{description}
\end{Desc}


Definition at line 529 of file ehrhart.c.\index{ehrhart.c@{ehrhart.c}!eequal@{eequal}}
\index{eequal@{eequal}!ehrhart.c@{ehrhart.c}}
\subsubsection{\setlength{\rightskip}{0pt plus 5cm}int eequal (evalue $\ast$ {\em e1}, evalue $\ast$ {\em e2})\hspace{0.3cm}{\tt  [static]}}\label{ehrhart_8c_a12}


\begin{Desc}
\item[Parameters: ]\par
\begin{description}
\item[{\em 
e1}]pointers to evalues \item[{\em 
e2}]pointers to evalues \end{description}
\end{Desc}
\begin{Desc}
\item[Returns: ]\par
1 (true) if they are equal, 0 (false) if not \end{Desc}


Definition at line 265 of file ehrhart.c.\index{ehrhart.c@{ehrhart.c}!emul@{emul}}
\index{emul@{emul}!ehrhart.c@{ehrhart.c}}
\subsubsection{\setlength{\rightskip}{0pt plus 5cm}void emul (evalue $\ast$ {\em e1}, evalue $\ast$ {\em e2}, evalue $\ast$ {\em res})\hspace{0.3cm}{\tt  [static]}}\label{ehrhart_8c_a14}


multiplies two evalues and puts the result in res.

\begin{Desc}
\item[Parameters: ]\par
\begin{description}
\item[{\em 
e1}]pointer to an evalue \item[{\em 
e2}]pointer to a constant evalue \item[{\em 
res}]pointer to result evalue = e1 $\ast$ e2 \end{description}
\end{Desc}


Definition at line 372 of file ehrhart.c.\index{ehrhart.c@{ehrhart.c}!free_evalue_refs@{free\_\-evalue\_\-refs}}
\index{free_evalue_refs@{free\_\-evalue\_\-refs}!ehrhart.c@{ehrhart.c}}
\subsubsection{\setlength{\rightskip}{0pt plus 5cm}void free\_\-evalue\_\-refs (evalue $\ast$ {\em e})}\label{ehrhart_8c_a8}


releases all memory referenced by e.

(recursive)  \begin{Desc}
\item[Parameters: ]\par
\begin{description}
\item[{\em 
e}]pointer to an evalue \end{description}
\end{Desc}


Definition at line 147 of file ehrhart.c.\index{ehrhart.c@{ehrhart.c}!new_enode@{new\_\-enode}}
\index{new_enode@{new\_\-enode}!ehrhart.c@{ehrhart.c}}
\subsubsection{\setlength{\rightskip}{0pt plus 5cm}enode$\ast$ new\_\-enode (enode\_\-type {\em type}, int {\em size}, int {\em pos})}\label{ehrhart_8c_a7}


EHRHART POLYNOMIAL SYMBOLIC ALGEBRA SYSTEM.

The newly allocated enode can be freed with a simple free(x)\begin{Desc}
\item[Parameters: ]\par
\begin{description}
\item[{\em 
type}]: enode type \item[{\em 
size}]: degree+1 for polynomial, period for periodic \item[{\em 
pos}]: 1..nb\_\-param, position of parameter  \end{description}
\end{Desc}
\begin{Desc}
\item[Returns: ]\par
a newly allocated enode \end{Desc}


Definition at line 121 of file ehrhart.c.\index{ehrhart.c@{ehrhart.c}!old_Polyhedron_Preprocess@{old\_\-Polyhedron\_\-Preprocess}}
\index{old_Polyhedron_Preprocess@{old\_\-Polyhedron\_\-Preprocess}!ehrhart.c@{ehrhart.c}}
\subsubsection{\setlength{\rightskip}{0pt plus 5cm}Polyhedron$\ast$ old\_\-Polyhedron\_\-Preprocess (Polyhedron $\ast$ {\em D}, Value {\em size}, unsigned {\em MAXRAYS})}\label{ehrhart_8c_a22}


This procedure adds additional constraints to D so that as each parameter is scanned, it will have a minimum of 'size' points If this is not possible, an empty polyhedron is returned.

\begin{Desc}
\item[Parameters: ]\par
\begin{description}
\item[{\em 
D}]\item[{\em 
size}]\item[{\em 
MAXRAYS}]\end{description}
\end{Desc}


Definition at line 1064 of file ehrhart.c.\index{ehrhart.c@{ehrhart.c}!print_enode@{print\_\-enode}}
\index{print_enode@{print\_\-enode}!ehrhart.c@{ehrhart.c}}
\subsubsection{\setlength{\rightskip}{0pt plus 5cm}void print\_\-enode (FILE $\ast$ {\em DST}, enode $\ast$ {\em p}, char $\ast$$\ast$ {\em pname})}\label{ehrhart_8c_a11}


prints the enode to DST.

\begin{Desc}
\item[Parameters: ]\par
\begin{description}
\item[{\em 
DST}]destination file  \item[{\em 
p}]pointer to enode to be printed  \item[{\em 
pname}]array of strings, name of the parameters \end{description}
\end{Desc}


Definition at line 220 of file ehrhart.c.\index{ehrhart.c@{ehrhart.c}!print_evalue@{print\_\-evalue}}
\index{print_evalue@{print\_\-evalue}!ehrhart.c@{ehrhart.c}}
\subsubsection{\setlength{\rightskip}{0pt plus 5cm}void print\_\-evalue (FILE $\ast$ {\em DST}, evalue $\ast$ {\em e}, char $\ast$$\ast$ {\em pname})}\label{ehrhart_8c_a10}


\begin{Desc}
\item[Parameters: ]\par
\begin{description}
\item[{\em 
DST}]destination file \item[{\em 
e}]pointer to evalue to be printed \end{description}
\end{Desc}


Definition at line 196 of file ehrhart.c.\index{ehrhart.c@{ehrhart.c}!reduce_evalue@{reduce\_\-evalue}}
\index{reduce_evalue@{reduce\_\-evalue}!ehrhart.c@{ehrhart.c}}
\subsubsection{\setlength{\rightskip}{0pt plus 5cm}void reduce\_\-evalue (evalue $\ast$ {\em e})\hspace{0.3cm}{\tt  [static]}}\label{ehrhart_8c_a13}


\begin{Desc}
\item[Parameters: ]\par
\begin{description}
\item[{\em 
e}]pointer to an evalue \end{description}
\end{Desc}


Definition at line 299 of file ehrhart.c.

\subsection{Variable Documentation}
\index{ehrhart.c@{ehrhart.c}!overflow_warning_flag@{overflow\_\-warning\_\-flag}}
\index{overflow_warning_flag@{overflow\_\-warning\_\-flag}!ehrhart.c@{ehrhart.c}}
\subsubsection{\setlength{\rightskip}{0pt plus 5cm}int overflow\_\-warning\_\-flag = 1}\label{ehrhart_8c_a5}


EPRINT : print results while computing the ehrhart polynomial.

this is done by default if you build the executable ehrhart. (If EMAIN is defined). Don't define EMAIN here, it is defined when necessary in the makefile.





Notice: you may however define EPRINT without defining EMAIN, but in this case, you have to initialize the global variable param\_\-name by calling Read\_\-Param\-Names before any call to ehrhart. This is NOT recommanded, unless you know what you do. EPRINT causes more debug messages to be printed. 

Definition at line 97 of file ehrhart.c.\index{ehrhart.c@{ehrhart.c}!param_name@{param\_\-name}}
\index{param_name@{param\_\-name}!ehrhart.c@{ehrhart.c}}
\subsubsection{\setlength{\rightskip}{0pt plus 5cm}char$\ast$$\ast$ param\_\-name}\label{ehrhart_8c_a6}




Definition at line 104 of file ehrhart.c.